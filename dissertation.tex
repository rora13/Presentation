\documentclass{beamer}
\mode<presentation>{
\usetheme{Warsaw}
\usecolortheme{sidebartab}
\setbeamercovered{transparent}
\usepackage{graphicx}    
\usepackage{amsmath}
\usepackage{multicol}
\usepackage{enumerate}
}
%\usepackage{ragged2e}
%\justifying
\usepackage[spanish]{babel}
\usepackage[utf8]{inputenc} 

\title{Conceptos Básicos de la Meteorología Sinóptica}
\author[\LaTeX{}]{Rodrigo Castillo Rodríguez}
\institute[CIGEFI]{
\textbf{Universidad de Costa Rica} \\
Escuela de Física \\
\textbf{Centro de Investigaciones Geofísicas} \\
CIGEFI\\

}
\date[6 2 2015]{$18$ de noviembre $2015$} 

\titlegraphic{\includegraphics[width=1.8 in]{figuras/escudocigefi.pdf}}


\newtheorem{Th1}{Reseña Historica} 
\newtheorem{Th2}{Definición}
%\newtheorem{Th3}{Sistemas de Presión Atmosférica [Verano Boreal]}
%\newtheorem{Th4}{Sistemas de Presión Atmosférica [Verano Austral]}
%\newtheorem{Th5}{}
%\newtheorem{Th6}{}
%\newtheorem{Th7}{}

\makeatletter
\def\beamer@tocaction@only#1{\only<.(1)>{\usebeamertemplate**{#1}}}
\define@key{beamertoc}{subsectionsonly}[]{\beamer@toc@subsectionstyle{show/only}\beamer@toc@subsubsectionstyle{show/shaded/hide}}

\begin{document}
\begin{frame}
\titlepage
\end{frame} 

\begin{frame}
%\frametitle{Esquema de trabajo}
\tableofcontents[subsectionsonly, pausesections] 
\end{frame}

\section{Circulación General Planetaria}
\subsection{Regímenes de Vientos}
\begin{frame}
\frametitle{Regímenes de Vientos}
%\begin{Th1}
\begin{figure}[!hbt]
   \centering
   \includegraphics[width=3.5 in]{figuras/celdas1.pdf}
\end{figure}
%\end{Th1}
\end{frame} 

\subsection{Celdas de Circulación Atmosférica}
\begin{frame}
\frametitle{Celdas de Circulación Atmosférica}
%\begin{Th2}
\begin{figure}[!hbt]
   \centering
   \includegraphics[width=3.2 in]{figuras/celdas2.pdf}
\end{figure}
%\end{Th2}
\end{frame} 

\subsection{Sistemas de Presión Atmosférica}
\begin{frame}
\frametitle{Sistemas de Presión Atmosférica [Verano Boreal]}
%\begin{Th3}
\begin{figure}[!hbt]
   \centering
   \includegraphics[width=3.3 in]{figuras/presion1.pdf}
\end{figure}
%\end{Th3}
\end{frame} 

\begin{frame}
\frametitle{Sistemas de Presión Atmosférica [Verano Austral]}
%\begin{Th4}
\begin{figure}[!hbt]
   \centering
   \includegraphics[width=3.3 in]{figuras/presion.pdf}
\end{figure}
%\end{Th4}
\end{frame} 

\section{Circulación Atmosférica Tropical}
\subsection{Zona de Convergencia Intertropical}
\begin{frame}
\frametitle{Zona de Convergencia Intertropical}
%\begin{Th1}
\begin{figure}[!hbt]
   \centering
   \includegraphics[width=3.65 in]{figuras/itcz.pdf}
\end{figure}
%\end{Th1}
\end{frame} 

\begin{frame}
\frametitle{Zona de Convergencia Intertropical}
%\begin{Th1}
\begin{figure}[!hbt]
   \centering
   \includegraphics[width=4.5 in]{figuras/itcz1.pdf}
\end{figure}
%\end{Th1}
\end{frame}

\subsection{Regímenes de Precipitación Monsónicos}
\begin{frame}
\frametitle{Regímenes de Precipitación Monsónicos}
%\begin{Th1}
\begin{figure}[!hbt]
   \centering
   \includegraphics[width=4.25 in]{figuras/monson.pdf}
\end{figure}
%\end{Th1}
\end{frame}

\section{Principal Modo de Variabilidad Climática}
\subsection{El Niño - Oscilación del Sur}
\begin{frame}
\frametitle{El Niño - Oscilación del Sur}
\begin{Th1}
El Niño, término originalmente usado por pescadores peruanos, definido inicialmente como un influjo de aguas costeras cálidas, ricas en nutrientes, provenientes del Golfo de Guayaquil; este influjo que anunciaba buenas capturas pesqueras usualmente ocurre en diciembre por lo cual los pescadores lo asociaron al \textcolor{blue}{nacimiento del niño Jesús (Navidad)}. Desgraciadamente, a este fenómeno de escala local se le superpone el influjo de aguas cálidas provenientes del océano Pacífico tropical, asociadas a la supresión de las surgencias costeras, pobres en nutrientes, lo cual reduce drásticamente la productividad pesquera en toda la región.
\end{Th1}
\end{frame} 

\begin{frame}
\frametitle{ENOS [Fase Neutra]}
%\begin{Th1}
\begin{figure}[!hbt]
   \centering
   \includegraphics[width=4.4 in]{figuras/neutro.pdf}
\end{figure}
%\end{Th1}
\end{frame}

\begin{frame}
\frametitle{ENOS [Fase El Niño]}
%\begin{Th1}
\begin{figure}[!hbt]
   \centering
   \includegraphics[width=4.4 in]{figuras/nino.pdf}
\end{figure}
%\end{Th1}
\end{frame}

\begin{frame}
\frametitle{ENOS [Fase La Niña]}
%\begin{Th1}
\begin{figure}[!hbt]
   \centering
   \includegraphics[width=4.4 in]{figuras/nina.pdf}
\end{figure}
%\end{Th1}
\end{frame}

\section{Distribución de Ciclones}
\subsection{Ciclones y Efecto Coriolis }
\begin{frame}
\frametitle{Efecto Coriolis}
\begin{Th2}
Es la fuerza aparente que desvía al viento o a cualquier parcela de fluido u objeto en
movimiento, a la derecha, en el hemisferio norte, o a la izquierda en el hemisferio sur.\\
El parámetro de coriolis está dada por la siguiente ecuación: 
\begin{equation}
f=2\Omega\sin\phi
\end{equation}
$\Omega$: velocidad angular terrestre \\
$\phi$: latitud
\end{Th2}
\end{frame} 

\begin{frame}
\frametitle{Distribución de Ciclones}
%\begin{Th1}
\begin{figure}[!hbt]
   \centering
   \includegraphics[width=4.5 in]{figuras/ciclones.pdf}
\end{figure}
%\end{Th1}
\end{frame}

\begin{frame}
\frametitle{Visión General}
%\begin{Th1}
\begin{figure}[!hbt]
   \centering
   \includegraphics[width=3.9 in]{figuras/escalas.pdf}
\end{figure}
%\end{Th1}
\end{frame}
 
\end{document} 
